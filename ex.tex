\documentclass[12pt,a4paper]{amsart}
\usepackage{amssymb}
\usepackage{listings}
\lstdefinelanguage{Macaulay2}
{morekeywords={for,from,if,do,while,apply,scan,print,method,help,viewHelp,table,exp,factor,needsPackage,ideal,det,determinan,res,coker,vars,Proj}, % TODO: use make-M2-symbols.m2
sensitive=false,
morecomment=[l]{--},
morecomment=[s]{-*}{*-},
morestring=[b]",
rulecolor=\color{blue!30},
}
\lstset{
  escapechar={`},
  aboveskip=0pt,
  belowskip=0pt,
  numbers=none,
  frame=leftline,
  framerule=1ex,
  framesep=1ex,
  xleftmargin=2ex,
  basicstyle={\ttfamily},
  columns=fixed,
  showstringspaces=false,
  commentstyle={\ttfamily\color{gray}},
  keywordstyle={\ttfamily\color{blue}},
  stringstyle={\ttfamily},
  fontadjust=true,
  basewidth={1.09ex},
  breaklines=false,
}
\usepackage[a4paper,left=2.25cm,right=2.25cm,top=3cm,bottom=3cm,headsep=1cm]{geometry}
\usepackage{tikz}%\usetikzlibrary{fit,positioning,matrix,calc,decorations.markings,angles,decorations.pathmorphing,decorations.pathreplacing}%,matrix,fit,shapes.geometric}
\title{Example file for M2inTeX}
\author{Paul Zinn-Justin}
\begin{document}
\maketitle

\section{Introduction}
some basic examples:
\begin{lstlisting}[language=Macaulay2]
R=QQ[x,y]; factor(x^3-y^3)
res coker vars R
OO_(Proj(R/(x^3-y^3)))^{1,2}
\end{lstlisting}
more:
\begin{lstlisting}[language=Macaulay2]
318/46
exp 3.73767
\end{lstlisting}
strings:
\begin{lstlisting}[language=Macaulay2]
"hehe"
\end{lstlisting}
and nets:
\begin{lstlisting}[language=Macaulay2]
"haha"||"hoho"
\end{lstlisting}
printing:
\begin{lstlisting}[language=Macaulay2]
for i from 1 to 8 do print(i^i)
\end{lstlisting}

\section{Help}
\begin{lstlisting}[language=Macaulay2]
help det
\end{lstlisting}

\section{Packages}
packages that have a {\tt tex} output will work:
\begin{lstlisting}[language=Macaulay2]
needsPackage "Posets";
booleanLattice 3
\end{lstlisting}

\section{Tricky examples}
\begin{lstlisting}[language=Macaulay2]
-- some tricky examples
\end{lstlisting}
A bunch of complicated cases: a multi-line example
\begin{lstlisting}[language=Macaulay2]
f = i -> (
-- that's dumb
i+1
)
\end{lstlisting}
and another weirder one:
\begin{lstlisting}[language=Macaulay2]
I=ideal 0; f = i -> (
i+1)
\end{lstlisting}
finally:
\begin{lstlisting}[language=Macaulay2]
a=1;b=2;
c=3;
\end{lstlisting}
That last one has no output.

\end{document}
